\documentclass[12pt,a4paper,portrait]{article}
%\setcounter{secnumdepth}{0}
\usepackage{gensymb}
\usepackage{pdflscape}
\usepackage{amsmath}
\usepackage{amssymb}
\usepackage{enumitem}
\usepackage{graphicx}
\usepackage{subcaption}
\usepackage{multirow}
\usepackage{sansmath}
\setcounter{secnumdepth}{4}
\renewcommand\paragraph{\@startsection{paragraph}{4}{\z@}%
	% display heading, like subsubsection
	{-3.25ex\@plus -1ex \@minus -.2ex}%
	{1.5ex \@plus .2ex}%
	{\normalfont\normalsize\bfseries}\\}
\usepackage{pst-eucl}
\usepackage{multicol}
\usepackage{csquotes}
% Coding
\usepackage{listings}
\setlength{\parindent}{0pt}
\usepackage[obeyspaces]{url}
% Better inline directory listings
\usepackage{xcolor}
\definecolor{light-gray}{gray}{0.95}
\newcommand{\code}[1]{\colorbox{light-gray}{\texttt{#1}}}
\usepackage{adjustbox}
\usepackage[UKenglish]{isodate}
\usepackage[UKenglish]{babel}
\usepackage{float}
\usepackage[T1]{fontenc}
\usepackage{setspace}
\usepackage{sectsty}
\usepackage{longtable}
\newenvironment{tightcenter}{%
	\setlength\topsep{0pt}
	\setlength\parskip{0pt}
	\begin{center}
	}{%
	\end{center}
}
\captionsetup{width=\textwidth}
\usepackage{mbenotes} % to print table notes!
\usepackage{alphalph} % For extended counters!
% usage: \tabnotemark[3]\cmsp\tabnotemark[4]
\usepackage[colorlinks=true,linkcolor=blue,urlcolor=black,bookmarksopen=true]{hyperref}
\sectionfont{%			            % Change font of \section command
	\usefont{OT1}{phv}{b}{n}%		% bch-b-n: CharterBT-Bold font
	\sectionrule{0pt}{0pt}{-5pt}{3pt}}
\subsectionfont{
	\usefont{OT1}{phv}{b}{n}}
\newcommand{\MyName}[1]{ % Name
	\usefont{OT1}{phv}{b}{n} \begin{center}of {\LARGE  #1}\end{center}
	\par \normalsize \normalfont}
\makeatletter
\newcommand\FirstWord[1]{\@firstword#1 \@nil}%
\newcommand\@firstword{}%
\newcommand\@removecomma{}%
\def\@firstword#1 #2\@nil{\@removecomma#1,\@nil}%
\def\@removecomma#1,#2\@nil{#1}
\makeatother

\newcommand{\MyTitle}[1]{ % Name
	\Huge \usefont{OT1}{phv}{b}{n} \begin{center}#1\end{center}
	\par \normalsize \normalfont}
\newcommand{\NewPart}[1]{\section*{\uppercase{#1}}}
\newcommand{\NewSubPart}[1]{\subsection*{\hspace{0.2cm}#1}}
\renewcommand{\baselinestretch}{1.05}
\usepackage[margin=0.2cm]{geometry}
\date{}
\setcounter{tocdepth}{4}

\title{Three body problem with the first body at the origin}
\author{Brenton Horne}

\begin{document}
	\maketitle
	
	In this document, we will derive the equations of motion for three bodies acting on each other solely through gravity. 
	\begin{figure}[H]
		\includegraphics[width=300px]{Three body problem.png}
	\end{figure}
	
	Here we have ignored the third dimension of space, as in many real-life problems bodies move largely in a 2D plane. We will be approaching finding the equations of motion for this system using the Euler-Lagrange equations:
	
	\begin{align}
		\dfrac{d}{dt}\dfrac{\partial \mathcal{L}}{\partial \dot{q}_i} - \dfrac{\partial \mathcal{L}}{\partial q_i} &= 0.\label{ELE}
	\end{align} 
	
	\tableofcontents
	
	\section{Velocities and distances}
	In this diagram, we have our axes centred on the first body (whose mass is $m_1$). Hence $x_1=y_1 = 0$ and $\dot{x}_1 = \dot{y}_1 = 0$. Hence the square of $m_2$ and $m_3$'s velocity relative to our axes is: 
	\begin{align*}
		v_2^2 &= \dot{x}_2^2 + \dot{y}_2^2 \\
		v_3^2 &= \dot{x}_3^2 + \dot{y}_3^2.
	\end{align*}
	
	Next we will calculate the distance between the centre of our masses as this is required to calculate the gravitational potential energy. Let $|\vec{r}_{ij}|$ denote the distance between bodies $i$ and $j$. 
	
	\begin{align*}
		\left|\vec{r}_{12}\right| &= \sqrt{x_2^2+y_2^2} \\
		\left|\vec{r}_{13}\right| &= \sqrt{x_3^2+y_3^2} \\
		\left|\vec{r}_{23}\right| &= \sqrt{(x_3-x_2)^2+(y_3-y_2)^2}.
	\end{align*}
	
	\section{Kinetic energies}
	
	The kinetic energies of the two bodies in motion are:
	
	\begin{align*}
		T_2 &= \dfrac{m_2}{2} v_2^2 \\
		&= \dfrac{m_2}{2} \left(\dot{x}_2^2 + \dot{y}_2^2\right). \\
		T_3 &= \dfrac{m_3}{2}v_3^2 \\
		&= \dfrac{m_3}{2} \left(\dot{x}_3^2 + \dot{y}_3^2\right).
	\end{align*}
	
	Hence the total kinetic energy of the system is:
	
	\begin{align*}
		T &= T_2 + T_3 \\
		&= \dfrac{m_2}{2} \left(\dot{x}_2^2 + \dot{y}_2^2\right) + \dfrac{m_3}{2} \left(\dot{x}_3^2 + \dot{y}_3^2\right)
	\end{align*}
	
	\section{Potential energy}
	The gravitational potential energy is:
	
	\begin{align*}
		V_{12} &= -\dfrac{Gm_1m_2}{|\vec{r}_{12}|} \\
		&= -\dfrac{Gm_1m_2}{\sqrt{x_2^2+y_2^2}}\\
		V_{13} &= -\dfrac{Gm_1m_3}{|\vec{r}_{13}|} \\
		&= -\dfrac{Gm_1 m_3}{\sqrt{x_3^2+y_3^2}} \\
		V_{23} &= -\dfrac{Gm_2m_3}{|\vec{r}_{23}|} \\
		&= -\dfrac{Gm_2 m_3}{\sqrt{(x_3-x_2)^2+(y_3-y_2)^2}} \\
		V &= \sum_{i,j} V_{ij} \\
		&= V_{12}+V_{13} +V_{23} \\
		&= -G\left[\dfrac{m_1m_2}{\sqrt{x_2^2+y_2^2}} + \dfrac{m_1 m_3}{\sqrt{x_3^2+y_3^2}} + \dfrac{m_2 m_3}{\sqrt{(x_3-x_2)^2+(y_3-y_2)^2}}\right].
	\end{align*}
	
	\section{Lagrangian}
	Hence the Lagrangian is:
	
	\begin{align*}
		\mathcal{L} &= T - V \\
		&= \dfrac{m_2}{2} \left(\dot{x}_2^2 + \dot{y}_2^2\right) + \dfrac{m_3}{2} \left(\dot{x}_3^2 + \dot{y}_3^2\right) + G\left[\dfrac{m_1m_2}{\sqrt{x_2^2+y_2^2}} + \dfrac{m_1 m_3}{\sqrt{x_3^2+y_3^2}} + \dfrac{m_2 m_3}{\sqrt{(x_3-x_2)^2+(y_3-y_2)^2}}\right].\\
	\end{align*}
	
	\section{Euler-Lagrange equations}
	\subsection{$x_2$}
	First we calculate the generalized momentum and its time derivative:
	\begin{align*}
		p_{x_2} &= \dfrac{\partial \mathcal{L}}{\partial \dot{x}_2} \\
		&= m_2 \dot{x}_2 \\
		\dot{p}_{x_2} &= m_2 \ddot{x}_2.
	\end{align*}
	Next we calculate:
	
	\begin{align*}
		F_{x_2} &= \dfrac{\partial \mathcal{L}}{\partial x_2} \\
		&= -\dfrac{Gm_1 m_2x_2}{\left(x_2^2+y_2^2\right)^{3/2}} -\dfrac{Gm_2m_3(x_2-x_3)}{\left((x_3-x_2)^2+(y_3-y_2)^2\right)^{3/2}}.
	\end{align*}
	
	Equation \eqref{ELE} hence becomes:
	
	\begin{align*}
		\dot{p}_{x_2} - F_{x_2} &= 0\\
		m_2 \ddot{x}_2 - \left(-\dfrac{Gm_1 m_2x_2}{\left(x_2^2+y_2^2\right)^{3/2}} -\dfrac{Gm_2m_3(x_2-x_3)}{\left((x_3-x_2)^2+(y_3-y_2)^2\right)^{3/2}}\right) &= 0 \\
		m_2 \ddot{x}_2 + \dfrac{Gm_1 m_2x_2}{\left(x_2^2+y_2^2\right)^{3/2}} +\dfrac{Gm_2m_3(x_2-x_3)}{\left((x_3-x_2)^2+(y_3-y_2)^2\right)^{3/2}} &= 0
	\end{align*}
	Dividing by $m_2$ yields and subtracting $\dfrac{Gm_1 m_2x_2}{\left(x_2^2+y_2^2\right)^{3/2}} +\dfrac{Gm_2m_3(x_2-x_3)}{\left((x_3-x_2)^2+(y_3-y_2)^2\right)^{3/2}}$ from both sides yields:
	
	\begin{align*}
		\ddot{x}_2 &= -\dfrac{Gm_1x_2}{\left(x_2^2+y_2^2\right)^{3/2}} -\dfrac{Gm_3(x_2-x_3)}{\left((x_3-x_2)^2+(y_3-y_2)^2\right)^{3/2}}.
	\end{align*}
	
	\subsection{$y_2$}
	Our equations are symmetric about $x$ and $y$ coordinates, so the equation for $y_2$ should be:
	
	\begin{align*}
		\ddot{y}_2 &= -\dfrac{Gm_1y_2}{\left(x_2^2+y_2^2\right)^{3/2}} -\dfrac{Gm_3(y_2-y_3)}{\left((x_3-x_2)^2+(y_3-y_2)^2\right)^{3/2}}
	\end{align*}
	
	\subsection{$x_3$}
	Our equations are also fairly symmetric about the body under study. So $x_3$'s equation should be similar to that of $x_2$:
	
	\begin{align*}
		\ddot{x}_3 &= -\dfrac{Gm_1x_3}{\left(x_3^2+y_3^2\right)^{3/2}} -\dfrac{Gm_2(x_3-x_2)}{\left((x_3-x_2)^2+(y_3-y_2)^2\right)^{3/2}}.
	\end{align*}
	
	\subsection{$y_3$}
	Utilizing symmetry again, yields:
	
	\begin{align*}
		\ddot{y}_3 &= -\dfrac{Gm_1y_3}{\left(x_3^2+y_3^2\right)^{3/2}} -\dfrac{Gm_2(y_3-y_2)}{\left((x_3-x_2)^2+(y_3-y_2)^2\right)^{3/2}}.
	\end{align*}
\end{document}