\documentclass[12pt,a4paper,openright]{article}
\usepackage{gensymb}
\usepackage{amsmath}
\usepackage{amssymb}
\usepackage{enumitem}
\usepackage{graphicx}
\usepackage{sansmath}
\usepackage{pst-eucl}
\usepackage{float}
\usepackage[numbered,framed]{matlab-prettifier}
\usepackage[T1]{fontenc}
\usepackage{setspace}
\usepackage{sectsty}
\usepackage[colorlinks=true,linkcolor=blue,urlcolor=black,bookmarksopen=true]{hyperref}
\setlength{\parindent}{0pt}
\renewcommand{\baselinestretch}{1.5}

\begin{document}
	The problem being solved in this directory is:
	\begin{align*}
	\ddot{\theta} = -\dfrac{g}{l} \cos{\theta}
	\end{align*}
	integrating both sides with respect to $\theta$ yields:
	\begin{align*}
		\dfrac{\dot{\theta}^2}{2} &= -\dfrac{g}{l} \sin{\theta} + C \\
		\dot{\theta} &= \pm \sqrt{-\dfrac{2g}{l} \sin{\theta} + C}\\
		\implies C &= \dot{\theta}_0^2 \\
		\therefore \hspace{0.1cm} \dot{\theta} &= \pm \sqrt{\dot{\theta}_0^2 - \dfrac{2g}{l} \sin{\theta}}
	\end{align*}
	
	where $\dot{\theta}_0$ is $\dot{\theta}$ when $\sin{\theta} = 0$ (therefore $\theta =n\pi$ where $n\in\mathbb{Z}$). $t$ can therefore be computed as:
	\begin{align*}
	t = \pm \int_{\theta_0}^{\theta_1} \dfrac{d\theta}{\sqrt{\dot{\theta}_0^2 - \dfrac{2g}{l} \sin{\theta}}}.
	\end{align*}
	
	In this repository, the initial conditions are:
	\begin{align*}
	\theta(t=0) = \dot{\theta}(t=0) = 0.
	\end{align*}
	In other words, the pendulum bob starts at the positive $x$ axis with zero velocity and moves solely under the influence of gravity. If we imagine a pendulum subject to these conditions, it becomes clear that theta will range from $-\pi$ (the bob being right on the negative $x$-axis) to $0$. If we wish to determine the period of $\theta$ (i.e. the value of $\chi$ such that $\theta(t+\chi) = \theta(t) \hspace{0.1cm}\forall t$), we must set $\theta_0=0$, $\theta_1 = -\pi$ and multiply our final result by two (as our result will only reflect how long it takes to go from the positive $x$ axis to the negative $x$ axis, not how long it will take to make the return trip)
	
	\begin{align*}
	\chi &= - 2\int_{0}^{-\pi} \dfrac{d\theta}{\sqrt{- \dfrac{2g}{l} \sin{\theta}}} \\
	  &= 2\int_{-\pi}^{0} \dfrac{d\theta}{\sqrt{- \dfrac{2g}{l} \sin{\theta}}}.
	\end{align*}
	
	Above we chose the negative on the $\pm$ sign because otherwise we will get a negative value for $t$, and we are choosing to keep time positive. It is impossible to solve this integral analytically and use it to express $\theta$ in terms of $t$, therefore we are reduced to using numerical methods to approximate $\theta$. The three numerical methods used in this directory are:
	
	\begin{itemize}
		\item \texttt{ode78} from the  ODE.jl Julia module.
		\item Runge-Kutta 4th order method. 
		\item The Newton-Kantorovich method to linearize the problem, and then a Chebyshev spectral method to approximate the solution to the linearized version of the problem.
	\end{itemize}
\end{document}