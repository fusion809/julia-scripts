\documentclass[12pt,a4paper,openright]{article}
\usepackage{gensymb}
\usepackage{amsmath}
\usepackage{amssymb}
\usepackage{enumitem}
\usepackage{graphicx}
\usepackage{sansmath}
\usepackage{pst-eucl}
\usepackage{float}
\usepackage[numbered,framed]{matlab-prettifier}
\usepackage[T1]{fontenc}
\usepackage{setspace}
\usepackage{sectsty}
\usepackage[colorlinks=true,linkcolor=blue,urlcolor=black,bookmarksopen=true]{hyperref}
\setlength{\parindent}{0pt}
\renewcommand{\baselinestretch}{1.5}

\begin{document}
	In scripts within this directory that start with \texttt{airy-}, Chebyshev spectral methods are used to numerically approximate the solution to the Sturm-Liouville problem (where $\lambda$ is an eigenvalue):

	\[
	- \dfrac{d^2 y}{dx^2} + xy = \lambda y, \hspace{0.1cm}\mathrm{on} \hspace{0.1cm} x\in[0,\infty].
	\]

	With the boundary conditions $y(0)=y(\infty)=0$. Both such scripts use the Chebyshev extrema grid that has been transformed to an approximation of the semi-infinite domain. 

	\texttt{Linear\_transformation.jl} uses a linear transformation of this grid to approximate the semi-infinite domain. The transformed domain goes from 0 to 870.
	
	\texttt{Rational\_transformation.jl} uses a rational transformation of this grid to approximate the semi-infinite domain, with $L=366$ and the transformation:
	
	\[
	x = L \left(\dfrac{1+x'}{1-x'}\right)
	\]

	where $x'$ is the variable of the Chebyshev extrema grid and $x$ is the variable of the transformed grid. I should clarify, however, that in \texttt{airy-rat.jl} the variable of the transformed grid is represented as $ y $ and the extrema grid is represented as $ x $. 
	
	The analytical solutions to this problem are:

	\[
	y_n(x) = a_n \mathrm{Ai}(x-\lambda_n)
	\]

	where the eigenvalues, $\lambda_n$, are the negative of the zeros of the Airy $\mathrm{Ai}(x)$ function and $a_n$ are arbitrary constants.

	Out of the two methods, the linear transformation yields the most accurate eigenvalues with almost 5,400 possible with $b~870$ and $N=10000$, whilst the rational transformation yields at best ~2,700 accurate eigenvalues with the same $N$ value (which is achieved at $L=366$). 
\end{document}
